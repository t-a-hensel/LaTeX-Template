%\documentclass[class=article, crop=false]{standalone}
%\usepackage[utf8]{inputenc}
\usepackage[T1]{fontenc}
\usepackage[english]{babel}
%\usepackage[iso]{date}
\usepackage{microtype} % optional, for aesthetics
\usepackage{csquotes}
\usepackage{fontawesome}
%%%%%%%%%%%%%%%%%%%%%%%%%%%%%%%%%%%%
%			Bibliography			%
%%%%%%%%%%%%%%%%%%%%%%%%%%%%%%%%%%%%
\usepackage[backend=biber,style=numeric-comp,sorting=none,natbib=true]{biblatex}
\addbibresource{./BIB/Bibliography.bib}

%%%%%%%%%%%%%%%%%%%%%%%%%%%%%%%%%%%%
%			Hyperref				%
%%%%%%%%%%%%%%%%%%%%%%%%%%%%%%%%%%%%
\usepackage{hyperref}
\usepackage{lastpage}
\hypersetup{
	breaklinks=false,
	citecolor=red,
	colorlinks=true,
	linkcolor=red,
	menucolor=black,
	pdfauthor={Thomas Arne Hensel},
	urlcolor=blue,
	bookmarks=true
}
%%%%%%%%%%%%%%%%%%%%%%%%%%%%%%%%%%%%
%				math				%
%%%%%%%%%%%%%%%%%%%%%%%%%%%%%%%%%%%%
\usepackage{amsmath}
\usepackage{amssymb}
\usepackage{amsthm}
\usepackage{commath}
\usepackage{siunitx}
\usepackage{eulervm}
%
%%%%%%%%%%%%%%%%%%%%%%%%%%%%%%%%%%%%
%		Figures and Plotting		%
%%%%%%%%%%%%%%%%%%%%%%%%%%%%%%%%%%%%
%
%\graphicspath{{SECTIONS/GRAPHICS/}}
\usepackage[dvipsnames]{xcolor}
\usepackage{graphicx}
\usepackage{caption,subcaption}%for subfigures
\usepackage{standalone}%to separately produce standalone figures
\usepackage{import}%to import figures later
%
\usepackage{tikz}%drawings like geometries
\usetikzlibrary{calc,matrix,positioning}
\usetikzlibrary{decorations.pathmorphing,patterns}
\usepackage{tikzscale}
%
\usepackage{pgfplots}
\pgfplotsset{
    ,compat=newest%1.12
    }
\usepgfplotslibrary{groupplots}
\usepgflibrary{patterns}
\usepackage{pgfplotstable}
%%%%%%%%%%%%%%%%%%%%%%%%%%%%%%%%%%%%
%				Tables				%
%%%%%%%%%%%%%%%%%%%%%%%%%%%%%%%%%%%%
\usepackage{multirow}
\usepackage{lscape}
\usepackage{pdflscape}
\usepackage{rotating}
\usepackage{tabularx}
\usepackage{booktabs}
%%%%%%%%%%%%%%%%%%%%%%%%%%%%%%%%%%%%
%			print git-hash			%
%%%%%%%%%%%%%%%%%%%%%%%%%%%%%%%%%%%%
\usepackage{etoolbox}
\newtoggle{submissionBuild}
\settoggle{submissionBuild}{false}
\nottoggle{submissionBuild}{%
	\usepackage{gitver}
	\usepackage{soul}
	\sethlcolor{green}
}{}
%%%%%%%%%%%%%%%%%%%%%%%%%%%%%%%%%%%%
%			Headings				%
%%%%%%%%%%%%%%%%%%%%%%%%%%%%%%%%%%%%
\usepackage{fancyhdr}
\setlength{\headheight}{15pt}

\pagestyle{fancy}
%\renewcommand{\chaptermark}[1]{ \markboth{#1}{} }
%\renewcommand{\sectionmark}[1]{ \markright{#1} }

\fancyhf{}
\fancyhead[LE,RO]{\footnotesize{p. \thepage\ / \pageref{LastPage}}}
\fancyhead[RE]{\emph{ \nouppercase{\leftmark}} }
\fancyhead[LO]{\emph{ \nouppercase{\rightmark}} }
\fancyfoot[CE,CO]{\iftoggle{submissionBuild}{}{%
  	\noindent{\emph{Revision}\/}: \hl{\mbox{\#\gitVer}}
	}}


\fancypagestyle{plain}{ %
  \fancyhf{} % remove everything
  \renewcommand{\headrulewidth}{0pt} % remove lines as well
  \renewcommand{\footrulewidth}{0pt}
}
% This will set fancy headings to the top of the page. The page number will be
% accompanied by the total number of pages. That way, you will know if any page is missing.
% If you do not want this for your document, you can just use``\pagestyle{plain}``.
%
\usepackage{todonotes}
%----------------------
%       Own definitions and macros
%----------------------
%
%
%----------------------
%       Annotation
%----------------------
%
\newcommand{\blue}[1]{\textcolor{blue}{#1}}%for blue comments
%\DeclareUnicodeCharacter{FFFD}{\blue{XXXX}}%to find false displayed characters, e.g. in Bib
%%%%%%%%%%%%%%%%%%%%%%%%%%%%%%%%%%%%
%				math				%
%%%%%%%%%%%%%%%%%%%%%%%%%%%%%%%%%%%%
\tikzset{
declare function={
        f(\z,\v,\x) = \z+\v*\x-0.5*\g*\x^2;
        }
}
%Define IFO-Parameters
\newcommand{\tmin}{0.7}
\newcommand{\dtmax}{\dtstart}
\newcommand{\dtstart}{1.0}
\newcommand{\T}{2.0}
\newcommand{\zmin}{3.0}
\newcommand{\dvstart}{0.8}
\newcommand{\vstart}{1.0}
\newcommand{\zmax}{9.0}
\newcommand{\g}{0.6}
\newcommand{\keff}{1.0}
%
% Word like operators.
\DeclareMathOperator{\acosh}{arcosh}
\DeclareMathOperator{\arcosh}{arcosh}
\DeclareMathOperator{\arcsinh}{arsinh}
\DeclareMathOperator{\arsinh}{arsinh}
\DeclareMathOperator{\asinh}{arsinh}
\DeclareMathOperator{\card}{card}
\DeclareMathOperator{\csch}{cshs}
\DeclareMathOperator{\diam}{diam}
\DeclareMathOperator{\sech}{sech}
\renewcommand{\Im}{\mathop{{}\mathrm{Im}}\nolimits}
\renewcommand{\Re}{\mathop{{}\mathrm{Re}}\nolimits}

% Fourier transform.
\DeclareMathOperator{\fourier}{\ensuremath{\mathcal{F}}}

% Roman versions of “e” and “i” to serve as Euler's number and the imaginary
% constant.
\newcommand{\ee}{\eup}
\newcommand{\eup}{\mathrm e}
\newcommand{\ii}{\iup}
\newcommand{\iup}{\mathrm i}

% Symbols for the various mathematical fields (natural numbers, integers,
% rational numbers, real numbers, complex numbers).
\newcommand{\C}{\ensuremath{\mathbb C}}
\newcommand{\N}{\ensuremath{\mathbb N}}
\newcommand{\Q}{\ensuremath{\mathbb Q}}
\newcommand{\R}{\ensuremath{\mathbb R}}
\newcommand{\Z}{\ensuremath{\mathbb Z}}

% Shape like operators.
\DeclareMathOperator{\dalambert}{\Box}
\DeclareMathOperator{\laplace}{\bigtriangleup}
\newcommand{\curl}{\vnabla \times}
\newcommand{\divergence}[1]{\inner{\vnabla}{#1}}
\newcommand{\vnabla}{\vec \nabla}

\newcommand{\half}{\frac 12}

% Unit vector (German „Einheitsvektor“).
\newcommand{\ev}{\hat{\vec e}}

% Scientific notation for large numbers.
\newcommand{\e}[1]{\cdot 10^{#1}}

% Mathematician's notation for the inner (scalar, dot) product.
\newcommand{\inner}[2]{\left\langle #1, #2 \right\rangle}

% Placeholders.
\newcommand{\emesswert}{\del{\messwert \pm \messwert}}
\newcommand{\fehlt}{\textcolor{darkred}{Hier fehlen noch Inhalte.}}
\newcommand{\messwert}{\textcolor{blue}{\square}}
\newcommand{\punkte}{\textcolor{white}{xxxxx}}

% Separator for equations on a single line.
\newcommand{\eqnsep}{,\quad}

% Quantum Mechanics
\newcommand{\bra}[1]{\left\langle #1 \right|}
\newcommand{\ket}[1]{\left| #1 \right\rangle}
\newcommand{\braket}[2]{\left\langle #1 \left. \vphantom{#1 #2} \right| #2 \right\rangle}
%\begin{document}
In this paper, the current limits for state-of-the-art precision experiments with atom interferometry were analyzed. A particular emphasis was put on the comparison of the statistical and systematic uncertainties between condensed and thermal ensembles. Three detailed study cases of a lab-based gravimeter, a space gradiometer and a satellite WEP-test were chosen to illustrate the limits of each regime.

Thermal sources benefit from a shorter cycle time and larger atom numbers compatible with experiments where moderate scale factors suffice or rapid readouts are required. This is, however, beneficial at short interferometry times only. When going beyond state-of-the-art, i.e. from drift times of a fraction of a second to a few seconds, this advantage is lost.
In our study cases, the scenarios utilizing BECs show - at equal integration time - the same shot-noise level and the magnitude of the mean-field effects is comparable to that of thermal sources. 
Moreover, the condensed sources benefit from a very large contrast (close to 1) when compared to their thermal counterparts. 
More dramatically, the WFA set an ultimate limit for thermal ensembles that would not be compatible with long interrogation times, which are required for advanced scenarios. 
For BEC ensembles, their compact sizes make this limit at least three orders of magnitude lower, highlighting their potential in the field of metrology.

Small scale distortions (few $\mu m$) of the optical beams~\cite{Bade2018}, not considered in this article, can hint to a disadvantage for the BEC samples by means of averaging effects for WFA.
However, the flexibility in tuning their initial size~\cite{Corgier2020} mitigates this effect and could bring them to starting sizes similar to thermal ensembles if necessary. 
This engineering of the BEC size allows for a distinct analysis of WFA with long and short periodicity~\cite{LouchetChauvet2011,Karcher2018}. 
This might be especially relevant on short time scales, i.e. for very small ensemble sizes. Their subsequent expansion could still be limited to a few mm thanks to the DKC technique. 
In consequence, a trade-off between the size-stretch-induced phase uncertainties, e.g. to balance the level of GGs or Coriolis systematics versus WFA effects is required. 
This appears to be feasible, especially if one considers gravity gradient compensation schemes as the ones in \cite{Loriani2020,Trimeche2019,Roura2017}.

Other considerations that are not reflected by our study would further consolidate the BEC choice. Indeed, we optimistically anticipate here that thermal ensembles could be collimated to the 80\,nK level and that the same level of efficiency in preparing, transporting and engineering of their the quantum states can be achieved as for BECs.
As a conclusion, thermal and BEC sources could equally be employed in relatively short interferometry times (a few hundred ms) for the same performance. With respect to longer times, BEC sources are clearly more advantageous since size-related systematic effects are several orders of magnitude smaller than those of thermal ensembles.

For the employment in IMUs, this study allows suggestsions of suitable sources only for a given scenario.
The various demonstrations of atom interferometers stably operated on chips with BECs point to a possible application as Quantum IMUs together with thermal atoms.
BECs could be used for longer interrogation times, as in rather stable scenarios, as in micro gravity, and thermal atoms for rapid readout in situations that are less stable and where a certain technical robustness is required, e.g. for cars.
Moreover, one could also think about the combination of both sources to gain highest precision and stability over long periods of time.
However, the study presents evidence that not only IMUs would benefit from atom interferometers operated with BECs, but a wide range of tests in fundamental physics and geodesic applications, too.
%
%\end{document}